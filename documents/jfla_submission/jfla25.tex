\documentclass{jflart}
% replace jflart with modjflart to have the preprint version, as stored on HAL

\usepackage[utf8]{inputenc}
\usepackage[T1]{fontenc}
\usepackage{alltt}

% Numéro et année des JFLAs visées par l'article, obligatoire.
\jfla{36}{2025}

\title{Chassez le naturel dans la formalisation des mathématiques}
% Un titre plus court, optionnel.
\titlerunning{Chassez le naturel}

% Auteurs, liste non abrégée.
\author[1]{Yves Bertot}
\author[1]{Thomas Portet}
% Une liste d'auteurs abrégée à utiliser à l'intérieur de l'article.
\authorrunning{Bertot et Portet}

% Affiliations des auteurs
\affil[1]{Centre Inria de l'Université Côte d'Azur, France}

% Une commande définie par l'utilisateur
\newcommand{\cmd}[1]{\texttt{\textbackslash {#1}}}
\newcommand{\mathcomp}{\textsc{Mathematical Components}}

\begin{document}

\maketitle

\begin{abstract}
  Nous nous intéressons à l'utilisation des systèmes de preuves basés sur la
  théorie des types pour l'enseignement des mathématiques.  Nous voulons
  remettre en question l'approche répandue qui consiste à introduire
  d'abord les nombres naturels, puis d'autres types de nombres.

  Nous explorons une approche où le type des nombres réels est le seul type
  utilisé pour toutes les définitions concernant des nombres.  Ceci nous oblige
  à reconsidérer les outils fournis pour définir des fonctions (en particulier
  la récursion), et pour calculer avec ces fonctions.
  L'une des caractéristiques de ce travail est de redonner de la place à
  la notion d'ensemble.

  Nous illustrons cette approche avec quelques exercices mêlant suites
  récurrentes et nombres réels, où la facilité à manipuler ensemble des nombres
  habituellement séparés dans des types distincts permet des expérimentations
  enrichissantes pour les étudiants.
\end{abstract}

\section{Introduction}

Il y a maintenant de nombreux systèmes de preuves interactifs
utilisables pour faire des preuves mathématiques.  Même si ces
systèmes de preuves ont été initialement conçus pour vérifier la
correction d'algorithmes et de logiciels, il est souvent pratique
d'inclure des capacités de raisonnement sur des objets mathématiques
pour justifier la correction de logiciels.  Par exemple, les
primitives cryptographiques peuvent reposer sur des propriétés
mathématiques d'objets remarquables comme des courbes elliptiques et
la spécification même de la correction de tels algorithmes peut
reposer sur le concept mathématique de probabilité.

Historiquement, pratiquement tous les systèmes de preuve basés sur la
théorie des types ont commencé
par décrire les entiers naturels, en spécifiant un type pour cette
catégorie de nombres, puis d'autres types de nombres sont ajoutés au
fur et à mesure que des concepts de plus en plus avancés sont fournis.
Dans la présentation des données fournies aux utilisateurs finaux, ces
nombres naturels apparaissent donc comme la première donnée
disponible.  C'est le cas dans le système Rocq
\cite{the_coq_development_team_2024_11551307} que nous utilisons dans
cette expérience.

Dans les systèmes basés sur la théorie des types avec des types
inductifs, comme Agda, Rocq, ou Lean, la situation est renforcée par
le fait que les entiers naturels se décrivent très bien comme un type
inductif.  Cette approche permet de disposer de capacités de calcul
relativement efficaces.

Lorsque l'on fait des mathématiques plus avancées, on est amené à
utiliser d'autres types de nombres, en particulier une grande partie
des mathématiques étudiées à l'école et dans les premières années
universitaires repose sur les nombres réels.  Si l'on veut couvrir
dans la bibliothèque d'un système de preuve des connaissances
correspondant à ce programme, on est naturellement amené à écrire des
formules où nombres naturels et nombre réels se côtoient.

L'essor des systèmes de preuve interactifs est tel que l'on peut
maintenant s'interroger sur leur apport dans l'enseignement des
mathématiques.  En particulier, l'étudiante en mathématique doit apprendre à
raisonner.  Pour raisonner, il faut savoir appliquer des
syllogismes, faire la différence entre les hypothèses et la conclusion
d'une phrase, comprendre des phrases énonçant l'existence d'un objet
ou des phrases énonçant qu'une propriété est universellement
satisfaite, et faire la différence entre ces deux types de phrases.

Pour de nombreux étudiants en mathématiques, le langage utilisé est
une langue étrangère.  Il faut pratiquer cette langue étrangère
fréquemment pour progresser.  En particulier, il faut apprendre à décrire
des raisonnements en suivant correctement les implications fournies par
les théorèmes fournis, et en faisant correctement les transformations
de formules permises par les égalités prouvées.  En fin de compte,
l'apprentissage est réussi si l'étudiante est capable de faire un
raisonnement qui sera considéré comme correct par une lectrice humaine.
En d'autres termes, il s'agit d'apprendre le langage qui permet de
s'inclure dans la communauté humaine des mathématiciens.

Un système de preuve interactif est justement conçu pour vérifier que toutes
les étapes de raisonnement sont effectuées correctement.  Il est donc
naturel de vouloir explorer l'utilisation d'un tel système de preuve
pour l'enseignement des mathématiques.  En revanche, il faut éviter
que ce nouvel outil impose une charge d'apprentissage supplémentaire,
contre-productive si l'étudiant passe trop de temps à
s'adapter au système de preuve mais n'a plus le temps d'apprendre les
concepts mathématiques de son cursus.  Il faut éviter que le langage
mathématique imposé par le système de preuve soit trop éloigné du
langage usuel pour le niveau d'apprentissage de l'étudiant.

Les nombres naturels, tels que manipulés dans les systèmes de preuve,
sont fournis comme un type inductif.  La théorie des types inductifs
fournit la possibilité de définir des fonctions par cas et par
récurrence de telle que sorte
que les opérations de base peuvent être définies, comme l'addition et
la multiplication et la soustraction.  Mais pour la soustraction
il apparaît une dissonance entre les
mathématiques telle qu'elles apparaissent dans le système de preuve et
dans la tradition humaine.  Le système de preuve impose
que la fonction soit totale, donc la conceptrice de la bibliothèque est
obligée de donner un sens aux formules où le deuxième argument est supérieur au
premier, comme \(3 - 5\).  Le choix fait
dans tous les systèmes de preuve est que le résultat est le nombre
\(0\), parce que la valeur réelle ou entière relative n'est pas un
entier naturel.

La tradition des mathématiques et d'associer à chaque fonction son
domaine de définition.  Lorsque l'on écrit une formule composée, on
doit expliciter le fait que toutes les fonctions sont bien utilisées
dans leur domaine de définition.  La tradition des mathématiques
formalisées en théorie des types est que toute fonction doit être
totale.  Les fonctions sont donc généralement prolongées avec une valeur
par défaut aux endroits où elles devraient être
non définies.  La modélisation reste fidèle dans la mesure où les
théorèmes qui permettent de raisonner sur la fonction utilisent
généralement l'appartenance des arguments au domaine de définition
comme hypothèse.  Ainsi, une mathématicienne utilisera naturellement
le raisonnement:
\[(m - n) + n = m\]
en reposant sur le fait que la formule de gauche a été validée (\(n
\leq m\)) plus tôt dans le discours.  En revanche, l'outil de preuve
fournit un théorème avec l'énoncé suivant:
\begin{verbatim}
forall m n, n <= m -> (m - n) + n = m
\end{verbatim}
Ainsi, la condition de bonne formation de la formule n'est pas
vérifiée a priori, mais uniquement au moment où l'on veut exploiter
les propriétés de cette fonction de soustraction.  Les bonnes
vérifications sont faites, mais à un moment différent.

Il reste que l'utilisatrice finale doit mémoriser le fait que
lorsqu'elle voit l'opération \(m - n\) dans une formule mathématique,
il ne s'agit pas de la soustraction usuelle, mais de cette
soustraction si \(n\) est plus petit que \(m\) et 0 sinon.

Le coût pour l'utilisatrice finale n'est pas seulement un coût de
mémorisation.  Les outils de démonstration (appelés des tactiques)
fournis dans le système de preuve sont également impactés par la
présence de cette représentation irrégulière de la soustraction.
Dans le système Rocq, il existe une tactique appelée \texttt{ring}, qui
est capable de prouver des égalités entre formules polynomiales, en
exploitant les propriétés des différents opérateurs.  Cette tactique
est très pratique et son comportement très facile à comprendre, sauf
qu'elle est ``bloquée'' en présence de soustractions sur les entiers
naturels.  Par exemple, la formule suivante est une formule évidente à
l'œil nu que cette tactique ne sait pas résoudre.
\[(n + m - (m + n)) * n = 0\]

Quand on avance dans les mathématiques, on est de toutes façons amené à
étudier des ensembles de nombres plus riches que l'ensemble des
entiers naturels.  Par exemple, si l'on considère la suite de
Fibonacci \({\mathcal F}\), définie par les équations suivantes:
\[{\mathcal F}(0) = 0 \qquad {\mathcal F}(1)= 1\qquad {\mathcal F}(n +
2) = {\mathcal F}(n) + {\mathcal F}(n + 1),\]
un résultat mathématique connu est donné par la formule suivante:
\[{\mathcal F} (n) = \frac{\phi ^n - ({\frac{-1}{\phi}}) ^
  n}{\sqrt{5}},\]
où \(\phi\) est le nombre d'or.
\[\phi = \frac{\sqrt{5} + 1}{2}\]
Alors que le membre gauche de l'égalité fait intervenir une fonction définie
seulement par récurrence sur les entiers naturels et en utilisant des
additions de nombres naturels, la formule de droite fait intervenir des
opérations qui ne sont pas stables pour les nombres naturels, les nombres
entiers relatifs, ou même les nombres rationnels.

Nous proposons de développer une libraire d'initiation à la preuve
mathématique dans laquelle on n'utilise qu'un seul type de nombres,
pour que les étudiants puissent travailler confortablement avec les
concepts mathématiques qu'ils doivent apprendre à maîtriser.  Ce
travail est complémentaire des efforts proposés par d'autres
chercheurs pour fournir un environnement où les étudiants peuvent
bénéficier d'une approche leur permettant de rédiger leurs preuves
dans un style de ``langue naturelle contrôlée'' parce que ces efforts
visent à améliorer le langage utilisé pour décrire les raisonnements
logiques, sans modifier le cadre de travail; les faits utilisés
restent proches de la théorie des types traditionnelle et reposent sur
plusieurs types de nombres.

Dans notre bibliothèque, les nombres naturels sont présents, mais
présentés comme un sous-ensemble des nombres réels.  De même nous
fournissons un ensemble des nombres entiers.  L'ensemble des
nombres naturels est
décrit par un prédicat inductif de type \({\mathbb R} \rightarrow
{\mathbb Prop}\).  Ceci nous permet de bénéficier directement des
fonctionnalités existantes pour les preuves par récurrence.  En
revanche, cette approche ne nous permet pas de définir des fonctions
récursives sur les entiers naturels.  Nous fournissons une commande
pour remplir cette lacune, et nous montrons sur quelques exemples que
cette commande permet de retrouver la capacité de raisonner sur des suites
récurrentes, un sujet qui est souvent abordé dans les premiers
cours de mathématiques.

Nous décrirons les grandes lignes d'implémentation des
différentes commandes que nous fournissons, nous donnons quelques exemples qui
illustrent les gains que permettent cette approche, en termes de
simplicité d'utilisation pour un étudiant découvrant à la fois les
mathématiques et les systèmes de preuve interactifs.

Les expériences décrites dans cet article sont visibles sur le site
suivant:

\url{https://github.com/ybertot/one_num_type/tree/jfla25}


\section{Décrire des sous-ensembles des nombres réels}
Dans le système de preuve, l'ensemble des nombres réels \(\mathbb R\)
est représenté par un type dénoté \texttt{R}.  Les opérations usuelles
sur les nombres sont représentées par des fonctions \texttt{+}, \texttt{-},
\texttt{*}, \texttt{/}.  Les constantes entières de type \texttt{R} s'écrivent
directement dans leur représentation décimale.  De cette manière, une
utilisatrice finale peut se contenter d'écrire la formule suivante
\texttt{3 - 5 * 2}.

Nous reviendrons plus tard sur le dispositif qui permet cette
utilisation de notations décimales, car il repose sur des nombres
entiers dans un type distinct, d'une façon dont l'utilisatrice ne
devrait pas avoir conscience.

L'approche usuelle pour définir un ensemble dans le système preuve
Rocq est d'identifier l'ensemble avec la propriété caractéristique de
cet ensemble.  Cette propriété caractéristique est une fonction, de
telle sorte que l'on écrit \texttt{Rnat n} pour exprimer que \texttt{n} est
dans l'ensemble \texttt{Rnat}.  Le système de preuve est basé sur la
théorie des types, donc cette fonction doit avoir un type de départ
(où elle prend ses arguments).  Ici, nous considérons des
sous-ensembles de \(\mathbb R\) donc la fonction a le type \texttt{R ->
  Prop}.

Les nombres naturels peuvent être décrits comme le sous-ensemble
minimal de \(\mathbb R\) qui contient \(0\) et qui est stable par
l'opération d'ajouter \(1\).  Le fait que l'ensemble considéré soit
minimal donne naturellement l'existence d'un principe de récurrence
qui s'énonce de la façon suivante: {\em toute propriété qui est vraie
  pour \(0\) et qui est vraie pour tout nombre de la forme \(n + 1\)
  si elle est déjà vraie pour \(n\) est vraie pour tous les entiers
  naturels}.

Dans les systèmes de preuve, une telle propriété définie par minimalité
peut généralement être construire à l'aide d'une définition inductive.
Pour les nombres naturels, la voici:
\begin{verbatim}
Inductive Rnat : R -> Prop :=
| Rnat0 : Rnat 0
| Rnat_suc : forall n, Rnat n -> Rnat (n + 1).
\end{verbatim}

Grâce à cette définition, on prouve aisément que la somme de deux
nombres naturels est un nombre naturel (par récurrence sur l'un des
deux arguments), de même pour le produit.  En revanche la soustraction
de deux nombres naturels ne satisfait pas nécessairement cette
propriété.  La différence de deux nombres est un entier relatif, et
cet entier relatif est un nombre naturel sous une condition d'ordre.

Ceci nous amène naturellement à considérer le sous-ensemble de \texttt{R}
qui contient les entiers relatifs.  Plusieurs façons de définir les
entiers positifs sont disponibles, nous avons choisi la suivante:
\begin{verbatim}
Inductive Rint : R -> Prop :=
| Rint1 : Rint 1
| Rint_sub : forall x y, Rint x -> Rint y -> Rint (x - y).
\end{verbatim}
Cette définition inductive permet de montrer que la somme de deux
nombres entiers est un nombre entiers, de même pour le produit et
la soustraction.

L'utilisatrice finale ne sera probablement pas amenée à utiliser le
principe de récurrence associé à la définition inductive \texttt{Rint}.
En revanche les preuves par récurrence usuelles d'un cours de
mathématiques seront habituellement faisable directement avec le
principe de récurrence associé avec \texttt{Rnat}.

Enfin, l'utilisatrice finale ne devra pas le voir,
mais pour nos développements, nous aurons besoin d'une
correspondance entre l'ensemble \texttt{Rnat} et les type inductifs
\texttt{nat} et \texttt{Z}.

La bibliothèque standard des nombres réels de Rocq fournit des fonctions
\texttt{INR} et \texttt{IZR} de type \texttt{nat -> R} et \texttt{Z -> R}.
Ces fonctions sont définies récursivement, de telle sorte que l'on peut
montrer facilement
que l'image de \texttt{INR} coïncide avec l'ensemble \texttt{Rnat}.  Les
énoncés sont les suivants:
\begin{verbatim}
Rnat_INR n : Rnat (INR n).

Rnat_exists_nat x {xnat : Rnat x} :
  exists n, x = IZR (Z.of_nat n).

Rnat_cst x : Rnat (ZR (Z.pos x)).
\end{verbatim}

Les différents théorèmes exprimant que l'ensemble des nombres naturels
est stable pour les opérations \texttt{+}, \texttt{*}, \texttt{Rabs},
sont déclarés comme des instances de classes de types.  Ainsi,
lorsque l'utilisatrice finale a besoin de vérifier qu'un nombre réel
satisfait la propriété d'être un nombre naturel, cette preuve peut souvent
être effectuée automatiquement.

Un des éléments remarquables de ces théorèmes de stabilité est le théorème
\texttt{Rnat\_cst}
qui exprime que toute constante numérique comme \(1\), \(3\), \(42\) est
appartien à l'ensemble \texttt{Rnat}.  Ce
théorème ne devrait pas être montré aux utilisateurs finaux, sous peine
d'avoir à expliquer l'existence de plusieurs types de nombres, mais il doit
être déclenché automatiquement chaque fois que l'ont veut vérifier qu'une
constante entière (de type \texttt{R}) est bien dans l'ensemble \texttt{Rnat}.
Ce théorème a pour énoncé:
\begin{verbatim}
Rnat_cst : forall p : positive, Rnat (IZR (Z.pos p))
\end{verbatim}
Ainsi, le nombre \texttt{42} de type \texttt{R} est affiché \texttt{42}, mais
est représenté en réalité par l'expression \texttt{IZR(Z.pos 42\%positive)}.
La sous expression \texttt{42\%positive} représente le nombre \texttt{42} dans
le type \texttt{positive}.
La preuve que le nombre réel \texttt{42} est bien élément de \texttt{Rnat}
est fournie par le théorème \texttt{IZR\_cst 42\%positive}.

La notation \texttt{42\%positive} mérite une explication.  L'analyse
syntaxique effectuée dans le système de preuve est paramétrable pour indiquer
que l'on veut lire données de différentes manières.
Ici, le suffixe \texttt{\%positive} permet d'indique que l'on veut
mentionner un nombre de type \texttt{\%positive}.  Nous verrons plus
tard que nous utiliserons également le suffixe \texttt{\%nat} pour
mentionner un nombre naturel de type \texttt{nat}.  L'existence de ces
notations obscucit notre présentation, mais c'est justement notre
objectif de faire disparaitre ce besoin de notations distinctes pour
différentes représentations du même nombre mathématique.

En revanche, le théorème qui permet de prouver qu'une soustraction retourne un
nombre naturel a la forme suivante:
\begin{verbatim}
Rnat_sub : forall n m, Rnat n -> Rnat m -> m <= n -> Rnat(n - m)
\end{verbatim}
Ce théorème doit souvent être invoqué interactivement.
\section{Définitions Récursives de Fonctions}
L'une des caractéristiques essentielles des types inductifs dans les
systèmes de preuve est qu'ils permettent de définir des fonctions
récursives.  En décidant de représenter tous les nombres utilisés dans
le discours mathématique par des nombres réels, nous perdons cette
caractéristique, parce que les nombres réels ne sont pas définis par
un type inductif (et ils ne peuvent pas l'être).

Pour remédier à cela, nous fournissons une commande qui permet de
définir des fonctions de \texttt{R} dans \texttt{R} de telle sorte que leur
valeur est définie précisément pour tous les nombres naturels par un
procédé récursif et qu'elles sont {\em indéfinies} ailleurs.  Cet usage
de valeurs
indéfinies est très similaire à ce qui se produit déjà pour l'inverse, dont
la valeur n'est pas définie en 0.  
L'inverse de 0 peut être manipulé comme un nombre réel, mais aucune
propriété ne peut être obtenue par preuve pour ce nombre.

Donnons un exemple de définition obtenue par notre commande:
\begin{verbatim}
Recursive (def factorial such that factorial 0 = 1 /\
  forall n, Rnat (n - 1) -> factorial n = n * factorial (n - 1)).
\end{verbatim}
À l'exécution de cette commande, deux nouveaux objets Rocq sont
engendrés, le premier est une fonction nommée \texttt{factorial} de
type \texttt{R -> R} et le
deuxième est un théorème appelé \texttt{factorial\_eqn} dont l'énoncé
est exactement la conjonction décrite ci-dessus.

Il est évident que la spécification donnée ci-dessus ne permet pas de
décrire la valeur de \texttt{factorial} pour l'entrée \texttt{0.5}.
En revanche, elle permet bien d'obtenir le résultat que
\texttt{factorial 3} est égal à \texttt{6}, en reposant sur plusieurs
réécritures avec le deuxième membre de la conjonction, sur les preuves que
\texttt{3 - 1}, \texttt{3 - 1 - 1}, \texttt{3 - 1 - 1 - 1} sont des
nombres naturels, et sur la vérification que \texttt{3 * (3 - 1) * (3 -
  1 - 1) = 6}.  Cette égalité est prouvée en une
seule étape par la tactique \texttt{ring}, parce que les nombres
considérés sont des nombres réels et la tactique \texttt{ring} n'a pas de
restriction vis-à-vis de la soustraction pour ce type de nombres.  La
tactique \texttt{ring} permet également de remplacer
\texttt{3 - 1}, \texttt{3 - 1 - 1}, \texttt{3 - 1 - 1 - 1} par
\texttt{2}, \texttt{1}, \texttt{0},
respectivement, et la preuve que ces nombres sont dans \texttt{Rnat} est
automatique.

La commande permet également de définir des suites récurrentes d'ordre
arbitrairement élevé (mais fini).  Les suites récurrentes les plus fréquentes,
où la valeur en \(n + 1\) est déterminée seulement en utilisant la valeur de
\(n\) et la valeur de la suite en \(n\), sont des suites d'ordre 1.  Lorsque
la valeur en \(n + k\) est déterminée seulement en utilisant la valeur de 
\(n\) et les valeurs déjà connues en \(n\), \(n + 1\), \dots \(n + (k - 1)\),
on dit que la suite est d'ordre \(k\).  Un exemple connu de suite d'ordre 2
est la suite de Fibonacci, où les valeurs en \(0\) et \(1\) sont \(0\) et \(1\)
respectivement, et la valeur en \(n + 2\) est la somme des deux
valeurs précédentes.

Avec notre commande, la suite de Fibonacci peut se définir de la façon
suivante (nous nommons la fonction \texttt{fib} pour gagner de la place).
\begin{verbatim}
Recursive (def fib such that
    fib 0 = 0 /\
    fib 1 = 1 /\
    forall n : R, Rnat (n - 2) ->
    fib n = fib (n - 2) + fib (n - 1)).
\end{verbatim}

\subsection{Techniques d'implémentation}
Ce que nous allons décrire dans cette section est la méthode que nous avons
employée pour concevoir notre commande de définition récursive.
L'utilisatrice finale n'a pas besoin de connaître ces détails: les
seuls éléments mis à sa disposition sont la fonction définie avec son
type, et le théorème qui collecte la spécification de comportement
sous forme d'une conjonction d'égalités, dont une est quantifiée
universellement sur un nombre dont un prédécesseur doit être un nombre
naturel.

Puisque l'utilisatrice finale ne voit pas ce code, nous avons le droit
d'utiliser les nombres naturels, et nous pouvons utiliser un outil de
méta-programmation pour manipuler la spécification
et construire une définition qui la satisfait.

\subsubsection{Forme des spécifications récursives}
Notre commande prend en argument un nom de fonction, disons \texttt{f},
et une
proposition qui spécifie le comportement de cette fonction.  Le nom de
la variable est donc lié dans la spécification.  Notre
commande n'accepte que des spécifications qui respectent une forme
syntaxique très précise: il doit s'agir d'une conjonction, dont les
membres sont de deux formes différentes.

La première forme spécifie
les valeurs pour les constantes \(0\), jusqu'à \(k - 1\).  La forme
syntaxique doit donc être \texttt{f i = V}.  Dans la suite, nous
appellerons ce nombre \(k\) {\em l'ordre de la récursion}.

La deuxième forme possible
pour un membre est une quantification universelle sur une variable,
disons \texttt{n}, le corps doit être une implication, dont la prémisse
est \texttt{Rnat (n - \(k\))}, la conclusion de cette implication doit être
une égalité de la forme \texttt{f n = B}.  La formule \texttt{B} doit être
une expression bien typée dans le contexte où \texttt{f} est une fonction
de type \texttt{R -> R} et \texttt{n} est un nombre de type \texttt{R}.
Nous appellerons cette formule \texttt{B} {\em le corps du pas récursif}.
Ce corps peut contenir des appels récursifs de
la forme \texttt{f (n - \(i\))} où \(i\) est une constant entière comprise
  entre \texttt{1} et \(k\).  Il doit y avoir un seul membre de la
  conjonction avec cette deuxième forme.

\subsubsection{Analyse des spécifications}
Un premier parcours récursif de la spécification, paramétré par
\texttt{f}, permet de retrouver l'ensemble des membres de la première forme
et de construire une liste de paires \((i, v_i)\) où les \(i\) sont des
constantes entières supérieures égale à 0 et les \(v_i\) sont les
termes représentant les valeurs spécifiées pour \(\texttt{f}~i\).  Cette
liste est triée, et on vérifie qu'il n'y a ni duplications ni manques.
La longueur de cette liste nous donne la valeur de \(k\).  A partir
de cette liste de paires triée, on construit une seconde liste qui ne
contient que les valeurs \(v_i\) dans le même ordre.  Nous appellerons
cette liste {\em la liste des valeurs initiales}.

Un deuxième parcours de la spécification permet de trouver le seul
membre qui est une quantification universelle sur une valeur \texttt{n}.
Un parcours récursif du membre droit \texttt{B} de l'égalité en conclusion de
ce membre droit, paramétré par \texttt{f} et \texttt{n}  permet de retrouver
l'ensemble des sous-termes de la forme \texttt{f (n - \(i\))} et de
construire la liste des valeurs \(i\), que nous appellerons {\em la liste des
décalages d'appels récursifs}.  Pour cette liste, il n'est
pas nécessaire de montrer l'absence de duplications, mais il est
nécessaire de vérifier que le maximum est inférieur ou égal à \(k\).

Si nous regardons la définition de \texttt{factorial} fournie dans la
section précédente, il y a un seul membre de la première forme
\texttt{factorial 0 = 1}, l'ordre de la récursion est 1,
la liste des valeurs initiales est \texttt{[1]}, le
corps du pas récursif est \texttt{n * factorial (n - 1)}, et la liste
des décalages d'appels récursifs est \texttt{[1]}.

Si nous regardons la définition de \texttt{fib} fournie dans la section
précédente, les membres de la première forme sont \texttt{fib 0 = 0} et
\texttt{fib 1 = 1}, l'ordre la récursion est 2,
la liste des valeurs initiales est \texttt{[0; 1]},
le corps du pas récursif est \texttt{fib (n - 2) + fib (n - 1)}, et
la liste des décalages d'appels récursifs est \texttt{[2,1]}.

\subsubsection{Utilisation du type \texttt{nat} et son récurseur}
L'implémentation repose sur le récurseur fourni pour le type
\texttt{nat}.  Nous allons définir une fonction de type \texttt{nat -> list
  R}, avec l'intention que la liste retournée pour l'entrée \texttt{n}
contienne les valeurs \texttt{f n}, \texttt{f (n + 1)} \dots
\texttt{f (n + \(k - 1\))}, où \(k\) est l'ordre de la récursion.

Le récurseur pour le type \texttt{nat} est une fonction appelée
\texttt{nat\_rect} avec le type suivant:
\begin{verbatim}
nat_rect : forall (A : nat -> Type) (Init : A 0)
  (step : forall n, A n -> A (n + 1)) (n : nat), A n
\end{verbatim}
Dans notre cas, nous utiliserons \texttt{nat\_rect} de façon non
dépendante, en instanciant \texttt{A} avec \texttt{fun \_ => list R},
de telle sorte la
valeurs \texttt{Init} devra être de type \texttt{list R} et \texttt{step} devra
être de type \texttt{nat -> list R -> list R}.  La fonction \texttt{step} a
pour objectif d'expliquer comment construire la valeur retournée pour
\texttt{n + 1}, lorsque l'on connaît la valeur de \texttt{n} (reçue en argument)
et la valeur déjà calculée pour \texttt{n} (également reçue en
argument).
La fonction \texttt{step} commencera
toujours par une abstraction de la forme \texttt{fun n (l : list R) => ...}.

Rappelons ici que la bibliothèque standard de Rocq contient une
fonction \texttt{nth} qui permet de retourner le \(n\)ième élément d'une
liste ou une valeur par défaut si la liste a moins de \(n\) éléments.  Il
y a aussi une fonction \texttt{INR} qui envoie tout nombre naturel vers
le nombre \texttt{R} qui lui correspond.

Pour la fonction factorielle, on voit que la valeur retournée pour
\texttt{n + 1} doit être la liste contenant un seul élément,
le produit de \texttt{n + 1} avec
la première valeur trouvée dans la liste en entrée, \texttt{nth 0\%nat l 0}.
\begin{verbatim}
(INR n + 1) * nth 0%nat l 0
\end{verbatim}
ceci doit être comparé avec le corps du pas récursif de la
spécification, qui est :
\begin{verbatim}
n * factorial (n - 1)
\end{verbatim}

Pour la suite de Fibonacci, , la liste retournée doit
contenir \texttt{fib (n + 1)} et \texttt{fib (n + 2)}.
La liste reçue en argument contient \texttt{fib
    n} et \texttt{fib (n + 1)}.  Le premier
élément de la liste retournée est tout simplement le deuxième élément
de la liste reçue en argument, ce qui s'écrit \texttt{nth 1\%nat l 0}.
Le deuxième élément de la liste retournée n'est pas présent dans la
liste reçue en argument, mais la spécification nous indique comment
calculer cette valeur en utilisant les valeurs présentes dans \texttt{l}.
La liste est donc de la forme suivante:
\begin{verbatim}
[nth 1%nat l 0; nth 0%nat l 0 + nth 1%nat l 0]
\end{verbatim}

Maintenant que nous avons vu ce que nous voulons sur deux exemples,
nous pouvons généraliser le traitement au cas d'une fonction
récurrente d'ordre \(k\) arbitraire.
\begin{itemize}
\item La fonction \texttt{step} doit
retourner une liste dont la longueur est l'ordre \(k\).
\item Les \(k-1\) premiers éléments de la liste retournée doivent
  être \texttt{nth 1\%nat l 0}, \texttt{nth 2\%nat l 0}, et ainsi de suite
jusqu'à \(k - 1\)
  (dans le cas de la fonction factorielle, il n'y a pas d'éléments de
  cette forme dans la liste).
\item Le dernier élément de la liste doit être une copie du corps du
  pas récursif, modifiée des deux manières suivantes.
\item Dans le corps du pas récursif, toute instance de \texttt{f (n -
  \(i\))} doit être remplacée par \texttt{nth \((k - i)\)\%nat l 0}.
\item Dans le corps du pas récursif, toute instance restante de \texttt{n} doit
  être remplacée par \texttt{INR n + \(k\)}.
\end{itemize}
Il faut noter qu'après les modifications au corps du pas de récursif,
il ne doit plus rester d'instance de la fonction en cours de
définition \texttt{f}.  Par
ailleurs, dans la spécification \texttt{n} a le type \texttt{R}, mais dans
la fonction \texttt{step}, \texttt{n} a le type \texttt{nat}, c'est pourquoi la
fonction \texttt{INR} doit être utilisée.  Il est aussi remarquable qu'à
l'endroit où une copie modifiée du corps du pas récursif est utilisé,
on est en train de
définir la valeur pour \texttt{f (n + \(k\))}.

\subsubsection{méta-programmation en \texttt{Elpi}}
Pour faire les analyses et la génération de code,
nous utilisons le langage de programmation
\texttt{Elpi}, fourni comme une extension de Rocq.  Ce langage est basé
sur \(\lambda\)-prolog, étendu avec des techniques de résolution de
contraintes.  Un point qui rend ce langage particulièrement concis
pour la conception de notre algorithme est que l'unité de base pour la
programmation est la règle logique, au lieu de la fonction, comme on a
l'habitude de le voir dans les langages de programmation fonctionnelle
avec construction filtrage.

Ainsi, une opération de remplacement, comme celles que nous décrivons
dans les quatrième et cinquième points de l'algorithme peut se décrire
en disant que l'on prend une fonction existante de copie de termes, sauf que les
instance du schéma à remplacer sont traitées par une règle prioritaire,
qui fait le remplacement au lieu de faire une simple copie à
l'identique.

Le texte suivant montre comment les remplacements sont effectués.  Le principe
mis en œuvre se résume de la façon suivante: faire un remplacement dans une
expression, c'est produire une copie légèremet modifiée de cette expression,
où le comportement du prédicat de copie est adapté pour reconnaitre les
expressions que l'on veut remplacer et mettre à leur place les expressions
cibles.  L'infrastructure du langage \texttt{Elpi} pour Rocq fournit déjà le
prédicat de copie (appelée \texttt{copy}) nous n'avons plus qu'à spécifier
une modification de ce prédicat en ajoutant temporairement
des règles de comportement pour ce prédicat, qui ont priorité sur les règles
existantes.  La
construction \texttt{=>} permet d'ajouter des règles temporaires.
La première règle temporaire est quantifiée universellement sur les variables
\texttt{I}, \texttt{I'}, \texttt{In} et \texttt{In'} grâce à la construction \texttt{pi},
Elle exprime qu'au lieu de copier, on va remplace les instances de termes
où la fonction \texttt{F} est
appliquée à \texttt{N} moins \texttt{I} par l'instance
correspondante construite avec \texttt{nth} \texttt{I'} et la liste de
valeurs
\texttt{L}.  Le calcul de \texttt{I'} est exprimé par les lignes de rang 4 à 6 dans
ce texte, et le point d'exclamation \texttt{!}, exprime que dès que le
motif est reconnu, il ne doit pas y avoir de retour arrière dans l'exécution.

La ligne de rang 7 décrit comment les instances restantes de \texttt{N} seront
remplacées par \texttt{N\_plus\_Order}, la valeur de
\texttt{N\_plus\_Order} ayant été calculée une fois pour toutes auparavant.
\begin{verbatim}
    (pi L \
      ((pi I I' In In' \ copy {{lp:F (lp:N - lp:I)}}
         {{nth lp:I' lp:L 0}} :-
         !,
          real_to_int I In,
          In' is Order - In,
          int_to_nat In' I'),
       copy N N_plus_Order) =>
    copy RHS (RHS' L)),
\end{verbatim}
Lorsque ce code est exécuté, une nouvelle constante \texttt{L} est créée pour
représentée temporairement une variable liée dont dépendra le résultat.
le terme dans la variable \texttt{RHS} est
copié dans un nouveau terme.  Alors que \texttt{RHS} ne contenait aucune
occurrence de la constante \texttt{L}, le nouveau terme en contient, donc
il existe une fonction \texttt{RHS'} tel que le terme obtenu soit égal
à \texttt{(RHS' L)}.  En conséquence, le terme \texttt{(RHS' L)} dépend de
la constante temporaire \texttt{L} mais la fonction \texttt{RHS'} n'en dépend
pas.  Fournir cette fonction \texttt{RHS'} est une opération
typique fournie par \(\lambda\)-prolog, et donc Elpi, et que l'on ne retrouve
pas dans les autres langages de programmation.  Ceci nous permet de gérer
de façon concise la notion de variable liée.

Ce fragment permet également d'illustrer la fonctionnalité de {\em
  quotation} qui permet d'écrire directement des expressions dans la
syntaxe usuelle de Rocq.  Ici, le terme
\begin{center}
\texttt{\{\{lp:F (lp:N -  lp:I)\}\}}
\end{center}
permet de décrire la représentation en Elpi d'un terme de
Rocq comportant l'application d'une fonction à une soustraction de
nombres réels.  Le préfixe \texttt{lp} dans \texttt{lp:N}
(pour \textit{lambda-prolog}) sert à indiquer que la variable \(\lambda\)-prolog
\texttt{N} va être instanciée avec la valeur qui se trouve à cet endroit dans
l'expression observée.

Lorsque la fonction basée sur \texttt{nat\_rect} est finalement
construite, on dispose d'une fonction de type \texttt{nat -> list R} et il
reste deux opérations à effectuer: prendre le premier élément du
résultat et fournir la bonne entrée.  Dans l'environnement de travail
pour l'utilisatrice finale, les seuls nombres sont de type \texttt{R},
donc nous avons besoin d'une fonction de type \texttt{R -> nat} qui
associe chaque nombre réel qui est un nombre naturel au nombre
correspondant de type \texttt{nat}.  Nous définissons une telle fonction
à l'aide de l'opérateur \texttt{epsilon} fourni par la bibliothèque de
logique classique.

Plus précisément, nous définissons une fonction \texttt{IRZ} de type
\texttt{R -> Z} qui associe tout nombre réel du sous-ensemble \texttt{Rint}
au nombre entier de type \texttt{Z} correspondant, puis nous composons
cette fonction avec une fonction de valeur absolue et de conversion
vers \texttt{nat} pour obtenir une fonction \texttt{IRN} de type \texttt{R ->
  nat} et le théorème de correction suivant:
\begin{verbatim}
INR_IRN x {xnat : Rnat x} : INR (IRN x) = x.
\end{verbatim}

Avec les contraintes que nous avons exprimées sur la spécification
d'une fonction récursive, il est assez facile de prouver que les
différents membres de la spécification sont satisfaits par la fonction
obtenue.  Nous avons écrit une tactique de quelques lignes pour cela
et la preuve est sauvegardée dans un théorème dont le nom est obtenu
en ajoutant le suffixe \texttt{\_eqn} au nom de la fonction.  Le code qui
décrit toute cette commande de définition consiste en 340 lignes dans le
langage Elpi (incluant des commentaires), plus la preuve de quelques
théorèmes annexes, plus une tactique écrite en 16 lignes du langage
\texttt{Ltac} pour prouver que la définition est bien satisfaite.

\subsection{Effectuer des preuves avec les fonctions récursives}
Pour effectuer des preuves sur les fonctions définies par notre
commande, nous avons deux outils: le principe de récurrence associé au
prédicat inductif \texttt{Rnat} et la spécification donnée par
l'utilisateur.  En effet notre commande fournit le théorème qui montre
que cette fonction satisfait cette spécification.

Pour illustrer l'utilisation de ces outils nous avons pris la
propriété déjà citée dans cet article, qui relie la suite de Fibonacci
et le nombre d'or.  Ce travail repose essentiellement sur les
tactiques \texttt{field} et \texttt{lra}, avec une petite difficulté causée
par le fait que \texttt{field} ne connaît pas la fonction de racine
carrée.  C'est donc l'utilisateur final qui doit trouver le moyen
d'éliminer la présence de cette racine dans les formules.

Nous avons donc prouvé l'énoncé suivant (où \texttt{phi'} est l'opposé
de l'inverse de \texttt{phi}).
\begin{verbatim}
Lemma Fibonacci_and_golden_ratio n:
    Rnat n -> fib n = (phi ^ n - phi' ^ n)/ sqrt 5.
\end{verbatim}
La preuve fait moins de 200 lignes écrites dans un style accessible à
des novices, avec des commentaires.
\subsection{Ajouter des capacités de calcul}
Pour les utilisateurs confirmés de Rocq, les nombres de types \texttt{nat} ou
\texttt{Z} ont l'avantage notable de pouvoir être utilisés pour faire des
calculs
expérimentaux, surtout les nombres entiers, dont la structure de données
exploite le codage binaire des nombres et permet donc la représentation
et le calcul de très grand nombres.  Cette capacité de calcul est perdue pour
les nombres réels.

Par exemple, la spécification de la fonction \texttt{fib} est
suffisamment expressive pour prouver la valeur de la fonction sur des
entrées constantes, mais le procédé devient rapidement impraticable.  Pour
permettre à l'utilisatrice finale de tester la fonction définie sur des
entrées assez grandes, nous avons fourni deux commandes.

La première de ces commandes s'appelle \texttt{R\_compute} et elle
permet de demander qu'un calcul soit effectué avec les opérations
usuelles.

La seconde s'appelle \texttt{mirror\_recursive\_definition}
et permet d'ajouter les fonctions récursives définies par notre
première commande parmi les fonctions connues par \texttt{R\_compute}.

Nous illustrons la première commande avec cet exemple.
\begin{verbatim}
R_compute (3 + (5 * 7)).
\end{verbatim}
Le résultat de cette commande est, comme on peut s'y attendre, 38.

Il est possible de donner un second argument à notre commande qui doit
être un nom.  Si ce second argument est donné, un théorème est créé
pour exprimer l'égalité entre l'expression donnée à calculer et le résultat
obtenu.  Ainsi, si l'on entre la
commande.
\begin{verbatim}
R_compute (3 + (5 * 7)) name1.
\end{verbatim}
Le théorème \texttt{name1} est créé par la commande, avec l'énoncé \texttt{3
  + 5 * 7 = 38}.

Dans l'état courant du développement, la commande \texttt{R\_compute}
ne fait que des calculs en nombres entiers, pour les opérations
d'addition, multiplication, soustraction, et opposé.  Le principe est
de faire les calculs dans \texttt{Z} au lieu de les faire dans \texttt{R},
en utilisant les correspondances prouvées déjà présentes dans la
bibliothèque de Rocq.  Ainsi la puissance de calcul des nombres
inductifs est mise en œuvre, mais l'utilisatrice finale ne donne en
entrée que des expressions réelles et ne reçoit en sortie que des
nombres réels.

La deuxième commande s'appelle \texttt{mirror\_recursive\_definition}.
L'objectif de cette commande est d'ajouter à la commande
\texttt{R\_compute} la possibilité de faire des calculs en utilisant
les fonctions récursives définies grâce à la commande présentée dans
la section précédente.  Ainsi, lorsque les fonctions
\texttt{factorial} et \texttt{fib} sont
déjà définies, la séquence de commande suivante se déroule avec succès,
elle affiche un nombre très grand et ajoute un théorème dans
l'environnement qui conserve une preuve du calcul.  La réponse vient
en moins d'une seconde (le calcul est instantané, mais la production
du théorème prend un peu plus de temps).
\begin{verbatim}
mirror_recursive_definition fib.

mirror_recursive_definition factorial.

R_compute (factorial 10 + fib 1000) fact10_fib1000_eq.
\end{verbatim}
En pratique, il y a quelques contraintes à respecter, parce que
l'ensemble des calculs doit pouvoir être accompagné d'une preuve.  La
contrainte la plus forte est que les fonctions récursives ne doivent
être appelées que sur des nombres naturels.  L'utilisatrice ne sait
pas nécessairement à priori si une expression est bien définie, mais
la commande \texttt{R\_compute} fournit un message d'erreur pour indiquer
qu'une des fonctions considérées est appelée avec un argument négatif
lorsque c'est le cas et aucun théorème n'est engendré.

\subsubsection{Réaliser la capacité de calcul}
La première chose qu'il est nécessaire de savoir, c'est que les
nombres réels dans Rocq ont déjà une relation privilégiée avec les
nombres entiers.  Toute constante entière qui s'écrit sous forme
décimale est déjà marquée comme l'application d'une fonction appelée
\texttt{IZR} sur une constante de type \texttt{Z}.  Cette fonction \texttt{IZR}
est donc déjà présente en plusieurs endroits dans la formule, mais
elle n'est jamais affichée (on parle de coercion implicite).

Pour effectuer le calcul, le principe est de transformer la formule à
calculer \(v\), qui est de type \texttt{R}, en une formule \(w\)
de type \texttt{Z}, en garantissant que \(v = \texttt{IZR}~w\).  Ceci est
fait de la façon suivante:
\begin{itemize}
\item Chaque instance du schéma \texttt{IZR \(v\)} est remplacée par \(v\),
\item Chaque instance d'une opération élémentaire sur les
réels (comme \texttt{+}, \texttt{*}, \texttt{-}, \texttt{Rabs}) est remplacée par
l'opération correspondante sur les entiers.
\end{itemize}
Une fois que la formule de type \texttt{Z} est obtenue, il suffit de demander
son calcul par Rocq (on parle aussi de réduction en forme normale).
Ce calcul retourne une constante entière \(z\) et le résultat
affiché est le nombre réel \texttt{IZR \(z\)}, mais en accord avec le
système de coercion implicite, le symbole \texttt{IZR} n'est pas affiché.

Pour justifier qu'une constante entière de la forma \(\texttt{IZR}~v\)
doit être traduite en \(v\), il suffit d'utiliser la réflexivité de
l'égalité, car dans ce cas la garantie à obtenir est simplement
\(\texttt{IZR}~v = \texttt{IZR}~v\).

En pratique, le remplacement des fonctions élémentaires sur \texttt{R} par
leur correspondant sur \texttt{Z} est fait en exploitant une table
comportant des enregistrements à trois champs.
\begin{enumerate}
\item le premier champ contient une fonction \(f\) sur \texttt{R}
\item le deuxième champ contient une fonction \(f_z\) sur \texttt{Z}
\item le troisième champ contient un théorème de {\em morphisme},
  exprimant que la fonction \texttt{IZR} est compatible avec les deux
  fonctions \(f\) et \(f_z\):
\begin{itemize}
\item si \(f\) est une fonction à un argument le théorème a l'énoncé
\[\forall x : \texttt{Z}, f~(\texttt{IZR}~x)=\texttt{IZR}(f_z~x)\]
\item si \(f\) est une fonction à deux arguments le théorème a l'énoncé
\[\forall x y : \texttt{Z}, f~(\texttt{IZR}~x)~(\texttt{IZR}~y)=\texttt{IZR}(f_z~x~y)\]
\end{itemize}
\end{enumerate}

Pour l'addition et la multiplication, les entrées dans la table ont
 la forme suivante:
\begin{verbatim}
  Rplus Z.add add_compute
  Rmult Z.mul mul_compute
\end{verbatim}

Pour utiliser les théorèmes de morphisme, nous les encapsulons dans deux
théorèmes qui facilitent l'usage récursif sans réécriture, un théorème
pour les fonctions à un argument et un théorème pour les fonctions à deux
arguments.  Nous montrons ici le théorème pour les fonctions à deux arguments.

\begin{verbatim}
private.IZR_map2 :
forall (opr : R -> R -> R) (opz : Z -> Z -> Z),
  (forall a b : Z, opr (IZR a) (IZR b) = IZR (opz a b)) ->
  forall (a b : R) (c d : Z), a = IZR c -> b = IZR d ->
  opr a b = IZR (opz c d)
\end{verbatim}

L'opération de traduction, que nous avons programmée à l'aide d'Elpi,
prend en entrée une expression \(E\) de type \texttt{R} et retourne une
expression \(E_z\) de type \texttt{Z} et un théorème dont l'énoncé est
\(E = \texttt{IZR}~E_z\).  La règle Elpi qui traite les fonctions à deux
arguments est la suivante:

\begin{verbatim}
translate_prf (app [F, A, B]) (app [F1, A1, B1])
  {{private.IZR_map2 lp:F lp:F1 lp:PFF1 lp:A lp:B lp:A1 lp:B1
           lp:PFRA lp:PFRB}}
  :-
  std.do! [
  thm_table F F1 PFF1,
  translate_prf A A1 PFRA,
  translate_prf B B1 PFRB
  ].
\end{verbatim}

Cette règle exprime que l'expression où \(F\) est appliquée à \(A\) et
\(B\) est traduite en l'expression où \(F1\) est appliquée à \(A1\) et
\(B1\), si \(F\) est en correspondance avec \(F1\), et si \(A\) et
\(B\) sont traduites respectivement en \(A1\) et \(B1\).  La preuve
que cette traduction est correcte est composée à l'aide de
 \texttt{private.IZR\_map2}, du théorème de morphisme pour \(F\) et \(F1\),
ici appelé \texttt{PFF1}, et des preuves pour \(A\) et \(B\).

Lorsque l'on affiche la preuve obtenue pour l'expression \texttt{3 + (5 *
  7)}, on voit donc le texte suivant.
\begin{verbatim}
th38 =
  private.IZR_map2 Rplus Z.add add_compute 3 (5 * 7) 3 (5 * 7)
   (eq_refl : 3 = 3)
   (private.IZR_map2 Rmult Z.mul mul_compute 5 7 5 7 
     (eq_refl : 5 = 5)
     (eq_refl : 7 = 7))
      : 3 + 5 * 7 = 38
\end{verbatim}
les nombres qui apparaissent en 6e et 7e argument de
\texttt{private.IZR\_map2} sont de type \texttt{Z}, mais comme le système de
preuve sait que des nombres entiers sont attendus à cet endroit, ils
sont affichés comme les nombres qui précèdent, qui sont de type
\texttt{R}.

\subsubsection{Réaliser le calcul pour les fonctions récursives}
La traduction des expressions peut également être effectuée sur le
corps de la fonction \texttt{step} utilisée pour le récurseur
\texttt{nat\_rect}.  Il y a quelques éléments supplémentaires utilisés
dans le corps de cette fonction: la variable \(n\) de type \texttt{nat}
et la variable \(l\) de type \texttt{list R}.  

Nous supposons que la valeur \(n\) est seulement utilisée dans les
appels récursifs et dans des
appels à \texttt{INR} afin que cette valeur numérique soit utilisée dans
opérations du type \texttt{R}.  Ces usages de \(n\) doivent être
remplacés par la valeur \(\texttt{Z.of\_nat}~n\).  Ces remplacements sont
justifiés par un théorème fourni dans la bibliothèque nommé
\texttt{INR\_IZR\_INZ}, avec l'énoncé suivant:
\begin{verbatim}
forall n : nat, INR n = IZR (Z.of_nat n)
\end{verbatim}

Dans la version actuelle de notre code, nous supposons que la variable \(l\) est
seulement utilisée dans des appels à la fonction \texttt{nth}.  Pour ces
appels, le premier argument qui est un entier naturel doit être laissé
inchangé, la liste \(l\) également, mais le dernier argument, qui
décrit la valeur par défaut qui serait utilisée si le premier
argument est supérieur à la longueur de la liste, doit être remplacé
par une valeur de type \texttt{Z}.

La correction du remplacement est justifiée par le théorème général
suivant sur le récurseur, qui se prouve par récurrence sur l'argument
entier naturel.  L'invariant de la récurrence est que les listes de
valeurs considérée à chaque étape sont les mêmes modulo application de
la fonction \texttt{IZR} pour convertir les valeurs numérique à chaque
rang dans la liste (exprimé par la fonction \texttt{map}.

\begin{verbatim}
private.nat_rect_list_IZR
     : forall (l0 : list Z) (l' : list R) 
         (f : nat -> list Z -> list Z)
         (f' : nat -> list R -> list R) (n : nat),
       l' = List.map IZR l0 ->
       (forall (k : nat) (lR : list R) (lZ : list Z),
        lR = List.map IZR lZ -> f' k lR = List.map IZR (f k lZ)) ->
       nat_rect (fun _ : nat => list R) l' f' n =
       List.map IZR (nat_rect (fun _ : nat => list Z) l0 f n)
\end{verbatim}
La cinquième ligne de cet énoncé exprime que l'une des conditions utilisée
est que la liste initiale de valeur est la même (modulo application
d'\texttt{IZR}).
La sixième et la septième ligne expriment que les fonctions
utilisées pour l'étape récursive transforment des listes
égales en des listes égales (à nouveau modulo application
d'\texttt{IZR}).

Une difficulté apparaît si la fonction récursive que l'on veut traiter
fait appel à une autre fonction récursive avec un argument qui n'est
pas de façon évidente un nombre naturel.  Il est possible que
l'utilisatrice qui définit cette fonction ait des connaissances
précises sur l'algorithme et puisse en déduire que la valeur passée
en argument soit toujours un nombre naturel.  Notre traducteur impose
que dans ce cas, un appel à la fonction \texttt{Rabs} soit inséré sur
l'argument de l'appel à cette autre fonction récursive.  Si la valeur
donnée en argument était déjà un nombre naturel, cette fonction
\texttt{Rabs} la laisse inchangée.

Les fonctions définies par \texttt{nat\_rect} sont essentiellement
conçues pour recevoir des arguments de type \texttt{nat}, mais les
fonctions utilisées dans les formules calculées doivent prendre en
entrée des nombres de \texttt{Z}.  Pour concilier les deux, les fonctions
que notre commande définit commencent par convertir le nombre entier en
un nombre naturel en faisant une valeur absolue.  Au
final, la correspondance entre le calcul en nombre réel et le
calcul en nombre entier n'est garantie que pour les entrées qui sont
des nombres naturels.
La commande \texttt{mirror\_recursive\_definition} créée une nouvelle
fonction dont le nom est obtenu en accolant le suffixe
\texttt{\_Z\_mirror} au nom de la fonction donnée en argument et un théorème
de correction dont le nom est obtenu en accolant le suffixe
\texttt{\_Z\_prf} avec l'énoncé de la forme suivante:
\begin{alltt}
forall x y, x = IZR y -> \(f\) (Rabs x) = IZR (\(f\)_Z_mirror y)
\end{alltt}

Ce théorème n'a pas la même forme que les théorème de morphismes
utilisés pour les opérations usuelles.  Premièrement il prend
directement la forme d'un théorème transformant une égalité sur les
entrées en une égalité sur les sorties.  Deuxièmement, il contient
un appel à la valeur absolue sur l'argument de la valeur réelle.
Puisque \texttt{x} est l'image de \texttt{y} par la fonction \texttt{IZR} nous
savons que c'est un entier et que \texttt{(Rabs x)} est un nombre
naturel.

Les fonction \texttt{\(f\)}, \texttt{\(f\)\_Z\_mirror}, et le théorème
\texttt{\(f\)\_Z\_prf} sont ajoutés dans une autre table de
correspondance.

Pour accepter quand même des formules dans lesquelles la fonction
récursive n'est pas appelée sur le résultat d'une valeur absolue, nous
utilisons le théorème de conversion suivant:
\begin{verbatim}
private.Rnat_Rabs : forall {f : R -> R} {fz : Z -> Z} :
   (forall (n : R) (z : Z), n = IZR z -> f (Rabs n) = IZR (fz z)) ->
   forall (n : R) (z : Z), Rnat n -> n = IZR z -> f n = IZR (fz z)
\end{verbatim}
Ce théorème a une hypothèse supplémentaire, qui demande de vérifier
que que l'argument de la fonction sur \texttt{R} est bien dans
\texttt{Rnat}.  Dans la commande \texttt{R\_compute} cette vérification peut
être faite à la volée en calculant simplement l'argument et en vérifiant
qu'il est positif.  Lorsque la preuve de correction est utilisée dans
\texttt{mirror\_recursive\_definition}, seule le théorème
{\(f\)\texttt{\_Z\_prf}} est utilisé.

Il y a également une tactique équivalente à \texttt{R\_compute} qui fait
le calcul et le remplacement d'une formule dans un but en cours de
preuve, sans sauvegarder le théorème d'égalité dans un théorème, mais
en incluant la preuve dans le terme en cours au moyen d'une réécriture.

\section{Séquences d'entiers naturels et grandes opérations}
Un idiome supplémentaire qui utilise des entiers naturels est celui
des séquences finies d'objets.  Par exemple, une famille linéaire de
vecteurs ou un vecteur de \texttt{R \^{ } n}.  Pour rendre ce type d'objets
pratique à utiliser, nous proposons d'introduire une fonction avec
une notation intuitive pour décrire une telle séquence, de fournir
des notations spécifiques pour une collection d'opérations des listes
vers les listes, en s'inspirant de la bibliothèque \mathcomp{}
\cite{MahboubiTassi2022} et
de reposer sur des {\em grandes opérations} \cite{BGOBP:BIG08}, sommes
itérées ou
produits itérés pour les opérations qui demanderaient normalement de
la programmation récursive.

A un niveau intermédiaire pour l'audience finale, ou pour un public
également formé à l'informatique, le cours pourrait également contenir
une introduction à la programmation récursive, mais seulement sur les
listes.  Il faudrait alors étendre l'outil de calcul pour fournir une
simulation calculatoire également pour les listes de nombres réels.
Commencer par la programmation récursive sur les listes présente un
avantage sur la programmation récursive sur les nombres naturels.

En effet,
même pour des étudiants en informatique, nous avons constaté que les novices
ont du mal à comprendre que l'on utilise le symbole \texttt{S} comme un
schéma de structure (et non comme l'opération d'addition avec le nombre 1), et
donc ils ont du mal à comprendre l'apparition de ce symbole dans le membre
gauche d'une règle de filtrage en programmation récursive.  Lorsque l'on
introduit la construction de filtrage sur les listes, il est plus naturel
de voir \texttt{(\_ :: \_)} comme un schéma de donnée, qui peut être utilisé
à la fois pour faire du filtrage et pour construire une liste.

\subsection{Séquences d'entiers naturels}
Nous proposons de fournir la fonction \texttt{Rseq} à deux arguments
\(n\) et \(m\)
pour désigner la séquence de nombres réels de longueur \(m\) et
contenant les nombres \(n\), \(n + 1\), \dots.
Comme pour les fonctions récursives, cette notation
n'est bien définie que si \(m\) est un nombre naturel.  Cette séquence
peut bien sûr être définie à l'aide d'un schéma récursif, comme nous le
faisons précédemment pour les suites récurrentes, mais il s'agit ici
d'une fonction produisant des listes.

Il est important de fournir deux théorèmes exprimant
le comportement calculatoire de cette fonctions récursive, plus des
théorèmes de décomposition, exprimés essentiellement à l'aide des
constructeurs du type des listes et la concaténation.
\begin{verbatim}
Rseq0 : forall n 0 : R, Rseq n 0 = nil

Rseq_suc : forall n m : R, Rnat m ->
  Rseq n (m + 1) = n :: Rseq (n + 1) m.
\end{verbatim}
A l'usage, le théorème \texttt{Rseq\_suc} n'est pas toujours pratique à
utiliser, car il demande de préparer la formule sur laquelle on
travaille pour faire apparaître l'expression de la forme \texttt{\(m\) +
  1} avant de permettre l'utilisation.  La forme suivante est souvent
plus facile à mettre en œuvre.
\begin{verbatim}
Rseq_suc' : forall n m : R, Rnat (m - 1) ->
  Rseq n m = n :: Rseq (n + 1) (m - 1).
\end{verbatim}
La présentation donnée jusqu'ici n'impose pas que le premier argument
de \texttt{Rseq} soit un entier naturel.  Il y a donc un gain par rapport
à l'approche reposant sur le type \texttt{nat}, puisque l'on peut
utiliser la même fonction et le même jeu de notations pour des listes de
nombres réels et pour des listes de nombres naturels.  A ce stade
préliminaire de nos expériences, ce gain est peut-être illusoire,
parce qu'il est plus difficile d'exprimer que si le premier argument
de \texttt{Rseq} est un entier naturel, alors tous les éléments de la
séquence sont des entiers naturels (une propriété qui était donnée
naturellement par le typage sur les entiers naturels).

\subsection{Grands opérateurs}
Les grands opérateurs apparaissent fréquemment dans les mathématiques
sous deux formes.  Soit il s'agit de présentations avec ellipses, où
la séquence de plusieurs nombres est présentée, mais séparés par un
opérateur binaire:
\[1 + \cdots + n  \qquad 1 ^ 2 + \cdots + n ^ 2 \]
Soit on utilise une notation avec variable liée et un symbole
conventionnel (parfois un version agrandie de l'opérateur binaire,
parfois un symbole ad hoc connu de tous), la notation mathématique
ressemble à ceci:
\[\sum_{i=0}^{n} f(i)\]
Dans \mathcomp{}, la deuxième présentation est préférée, en utilisant
des notations qui s'inspirent de ce que l'on doit écrire en \LaTeX{}
pour obtenir la présentation ci-dessus.  Nous voulons nous inspirer de
\mathcomp{}, mais en réduisant le nombre de possibilités, pour réduire
aussi les sources de confusion pour les novices.

Nous proposons de réduire l'usage des grands opérateurs aux deux
opérateurs d'addition et multiplication (notation \cmd{sum} ou \cmd{prod})
et au seul cas où
l'indice parcourt les entiers compris entre \({m}\) et \({n - 1}\),
en utilisant la notation \texttt{\(m\) <= \(i\) < \(n\)}.  Cette
notation exprime que le domaine sur lequel s'applique l'opération
itérée est vide si \(n \leq m\).  Le choix d'utiliser une comparaison
stricte fait encore l'objet de réflexions.  Il provient de la bibliothèque
\mathcomp{}, ou il est motivé par des questions de régularité dans les calculs,
mais les mathématiciens utilisent
peut-être plus naturellement des comparaisons au sens large.

Pour les preuves, nous pouvons utiliser les théorèmes de base qui
expriment le comportement récursif des opérations itérées
en enlevant de la somme ou du produit itéré des valeurs en commençant
par l'une ou l'autre des extrémités.  Dans la bibliothèque \mathcomp{}
qui nous sert d'inspiration, ces théorèmes sont génériques et s'appliquent
sur des types quelconques.  Dans notre bibliothèque destinée à des novices,
nous donnons l'implémentation pour les opérations usuelles.  Ainsi, la
somme d'une collection d'entiers s'écrit
\texttt{\cmd{}sum\_(a <= i < b) ...} et
le théorème suivant permet de montrer qu'une telle somme peut se décomposer
en l'addition de la somme d'une collection plus petite et de la dernière
valeur: on enlève la valeur prise à droite, d'où l'usage du suffixe \texttt{r}
dans le nom du théorème.  Nous avons également un théorème \texttt{sum0}
qui exprime que la somme d'une collection vide d'entiers est nulle.


\begin{verbatim}
sum_recr (E : R -> R) (a b : R) :
  Rnat (b - a) -> a < b ->
  \sum_(a <= i < b) E i =
   (\sum_(a <= i < (b - 1)) E i) + E (b - 1).

sum0
     : forall (E : R -> R) (a : R), \sum_(a <= i < a) E i = 0
\end{verbatim}

Nous avons testé ces définitions et théorèmes sur des preuves élémentaires.
\[\sum_{0 \leq i < n} i = \frac{n (n - 1)}{2} \qquad
\sum_{0 \leq i < n} i ^2 = \frac{n (n - 1) (2 n - 1)}{6} \qquad
n! = \prod_{1 \leq i < n + 1} i\]
Dans les trois cas, l'avantage notable par rapport aux preuves
utilisant le type des entiers naturels est que les preuves d'égalité
sont généralement résolues par les tactiques \texttt{ring} ou \texttt{field},
cette dernière étant nécessaire
lorsqu'une division apparaît.  Le prix à payer est la nécessité de
prouver des conditions annexes, sous forme d'appartenance à l'ensemble
\texttt{Rnat} ou des comparaisons entre entiers.  Pour résoudre ces
conditions, il est important d'exploiter le fait que tout nombre
naturel est positif ou nul et le fait qu'un nombre entier
strictement inférieur à un nombre entier \(n\) est inférieur ou égal au
prédécesseur de \(n\).

La bibliothèque \mathcomp{} est une source très riche de lemmes
supplémentaires pour effectuer des raisonnements généraux sur les
opérations itérées.  En particulier, elle permet de faire des preuves
où l'on exploite l'information que dans la somme itérée, l'index
d'itération est compris entre les bornes de la somme.  Dans nos
expériences préliminaires, nous n'avons pas encore mis au point un
théorème exprimant cette propriété.


\section{La fonction binomiale}
La fonction binomiale peut être présentée alternativement comme une fraction
d'expressions obtenues avec des factorielles ou comme une
fonction récursive sur les entiers naturels.  La bibliothèque
standard des réels utilise la définition suivante:
\begin{verbatim}
C =
fun n p : nat => INR (fact n) / (INR (fact p) * INR (fact (n - p)))
\end{verbatim}
Cette définition repose sur la définition récursive dans le type 
\texttt{nat} de la fonction factorielle (ici appelée \texttt{fact}, puis
une coercion dans les nombres réels, puis une division dans le type
des nombres réels.

La bibliothèque \mathcomp{} utilise plutôt la définition récursive suivante:
\begin{verbatim}
Fixpoint binomial n m :=
  match n, m with
  | S n', S m' => binomial n' m + binomial n' m'
  | _, 0 => 1
  | 0, S _ => 0
  end.
\end{verbatim}
Le nombre \texttt{binomial n m} est donc une valeur entier naturel, susceptible
d'être injectée dans n'importe quel anneau ou corps grâce à une
fonction générique qui joue un rôle similaire à \texttt{INR}.

La motivation pour montrer ces deux définitions est de remarquer
qu'elles ne coïncident pas lorsque le premier argument est strictement
inférieur au second.   Dans la définition de \texttt{C},  dans le cas où
\texttt{n}  est strictement inférieur à \texttt{p}, l'expression \texttt{n -
  p)} est égale à 0, la
factorielle de la soustraction est égale à 1, et le résultat est un
nombre rationnel strictement compris entre 0 et 1.  Dans la definition
de binomial, on peut démontrer par récurrence que la valeur retournée
est 0 (c'est le théorème \texttt{bin\_small} dans la bibliothèque
\mathcomp{}).

S'il y a une différence, c'est que l'endroit où les deux définitions
divergent n'est pas significatif.

La page Wikipedia en anglais (en octobre 2024) présente une troisième
définition, plus
efficace que les deux autres (mais pas encore optimale si \(k\) est proche
de \(n\)).
\[\frac{n \times (n - 1) \cdots \times (n - k + 1)}
{k \times (k - 1) \cdots \times 1}\]

Nous pouvons définir cette fonction dans notre librairie de la façon
proposée par cette page:
\begin{verbatim}
Definition choose (n k : R) :=
  (\prod_(m - k + 1 <= i < n + 1) i) / factorial k
\end{verbatim}

Pour montrer que la valeur est bien entière, nous pouvons commencer
par réconcilier la définition avec les deux autres, du moins pour les
arguments qui le permettent.
\begin{verbatim}
choose_n_factorial : forall n m, Rnat n -> Rnat m, m <= n -> 
  chose n k = factorial n / (factorial (n - k) * factorial k)

choose_n_0 : forall n, Rnat n -> choose n 0 = 1

chose_n_n : forall n, Rnat n -> chose n n = 1

choose_suc k n : Rnat k -> Rnat n -> k < n ->
   choose (k + 1) (n + 1) = choose k n + choose (k + 1) n
\end{verbatim}
Le premier lemme met en relation notre définition avec
la définition basée sur trois factorielles.  Les trois autres
lemmes sont très similaires aux règle de réduction donnée pour
la définition de la fonction \texttt{binomial} par récursion sur le type
inductif \texttt{nat} comme on le trouve dans la bibliothèque
\mathcomp{}.

La multiplicité des points de vue sur la fonction binomiale permet de
faire de nombreux exercices sur cette fonction.  En particulier, nous
avons fait quelques essais sur des exercices suggérés par Pierre
Rousselin et ses collègues \cite{RousselinPF24}.
\section{Travaux similaires et reliés}
D'autres efforts existent pour rendre les systèmes de preuves basés
sur la théorie des types pratiques pour l'enseignement.  Waterproof
\cite{Wemmenhove_2024}
utilise Rocq pour former les étudiants à écrire des preuves qui sont
acceptables pour la communication avec des humains, à commencer par
leur professeurs de mathématiques.  De façon similaire, Massot
\cite{massot:LIPIcs.ITP.2024.27}
propose
de modifier les tactiques de Lean pour que les utilisateurs emploient
un langage proche du langage naturel pour intéragir avec le système de
preuve.

Notre travail est complémentaire de ces travaux.  En rendant
l'intéraction avec le type de nombre plus automatique, il permet de
faciliter leurs efforts pour produire un texte mathématique qui semble
naturel à la lecture.  En revanche, nous avons volontairement limité
notre exploration du langage de tactique, justement parce que notre
intention est de bénéficier de leurs approches.

Pierre Rousselin et ses collègues développent
également des jeux d'exercices pour apprendre l'utilisation des
systèmes formelles dans les premières années universitaire
\cite{RousselinPF24}.  Les
exercices qu'ils ont produit nous ont servi d'inspiration à plusieurs
reprises.

Les systèmes de preuve Mizar \cite{Mizar-beyond2015} et PVS
\cite{PVS11} fournissent aussi
la possibilité d'utiliser un seul type de nombre, mais la question ne
se pose pas avec la même acuité car les frontières entre les types ne
sont pas aussi contraignantes que dans Rocq.

\section{Conclusion}
{\em Chassez le naturel, il revient au galop}.  De nombreuses commandes
et tactiques de Rocq sont conçues dans l'esprit que les nombres naturels
sont toujours là à disposition des utilisateurs.  Par exemple, la tactique
\texttt{ring\_simplify} crée spontanément des puissances où l'exposant est un
nombre naturel.

Le travail que nous suggérons et commençons dans cet article est de
faire l'inventaire de tous les endroits où le type des entiers
naturels est utile, et d'enlever cet usage de la présentation des
mathématiques.  Cela ne veut pas dire
que l'on veut faire disparaître le type des entiers naturels de la
construction des objets sur lesquels nous reposons, mais que chaque
concept présenté à l'utilisatrice finale doit utiliser les nombres
réels et le prédicat \texttt{Rnat} comme interface.

Plus tard dans le déroulement des cours, il est probablement
judicieux de réintroduire les
nombres naturels et les autres types inductifs de nombres dans
l'espace de travail, de façon à bénéficier directement des bénéfices
que chacun peut apporter.  Le point de vue de cet article est surtout
d'éviter de surcharger l'utilisateur dans ses premiers pas
d'apprentissage des mathématiques à l'aide des outils de preuve
formelle.

Pour l'objectif d'enseignement, il est important d'ajouter
plus d'outils pour faciliter les preuves d'appartenance aux
sous-ensembles choisis.  Il apparait également qu'un travail plus
systématique devrait être fait pour associer à des fonctions un
domaine de définition et systématiser les preuves pour vérifier qu'une
fonction est bien appelée sur un argument dans son domaine de
définition.

\bibliographystyle{alpha-fr}
\bibliography{bib}

\end{document}

% LocalWords:  Agda Rocq Lean texttt sqrt
% LocalWords:  qquad mathbb rightarrow Prop Rnat em Rint nat INR eqn
% LocalWords:  factorial fib subsection récurseur list rect Init nth
% LocalWords:  itemize Elpi -prolog lp IRZ IRN IZR enumerate forall
