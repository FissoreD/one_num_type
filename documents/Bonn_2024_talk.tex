\documentclass[compress]{beamer}
\usepackage[latin1]{inputenc}
\usepackage{cancel}
\usepackage{alltt}
\newdimen\topcrop
\topcrop=10cm  %alternatively 8cm if the pdf inclusions are in letter format
\newdimen\topcropBezier
\topcropBezier=19cm %alternatively 16cm if the inclusions are in letter format

\setbeamertemplate{footline}[frame number]
\title{A single number type for Math education\\in Type Theory}
\author{Yves Bertot}
\date{June 2024}
\mode<presentation>
\begin{document}

\maketitle
\begin{frame}
\frametitle{The context}
\begin{itemize}
\item Type theory based theorem provers are increasingly used for mathematics
\item Mathematical Components, Mathlib, Isabelle's AFP, etc,
are directed at experts
\item This talks concentrates on a different mathematical audience: early
  learners of mathematics
\begin{itemize}
\item Strong inspiration: Waterproof (and similar experiments with controled natural language)
\item The language is not the only problem, the material may also be an issue
\item Strong contention against the natural numbers
\end{itemize}
\item Typing helps young mathematicians, but not the type of natural numbers
\end{itemize}
\end{frame}
\begin{frame}
\frametitle{Issues with the natural numbers}
\begin{itemize}
\item Positive sides
\begin{itemize}
\item An inductive type
\item computation by reduction (faster than rewriting)
\item Proof by induction as instance of a general scheme
\item Recusive definitions are quite natural
\end{itemize}
\item Negative sides
\begin{itemize}
\item Subtraction is odd: the value of \(3-5\) is counterintuitive
\item The status of function/constructor {\tt S} is difficult to grasp.
\item In Coq, {\tt S 4} and {\tt 5} and interchangeable,
   but\\ {\tt S x} and {\tt x + 1} are not
\item Inductive types distort the notion of {\sl computation}
\item Too much cognitive load for struggling students
\end{itemize}
\end{itemize}
\end{frame}
\begin{frame}
\frametitle{Numbers in the mind of math beginners}
\begin{itemize}
\item Starting at age 12, kids probably know about integer, rational, and
real numbers
\item \(3 - 5\) exists as a number, it is not 0
\item Computing \(127 - 42\) yields a natural number, \(3 - 5\) an integer,
and \(1 / 3\) a rational
\item \(42 / 6\) yields a natural number
\item These perception are {\em right}, respecting them is time efficient
\end{itemize}
\end{frame}
\begin{frame}
\frametitle{Proposal}
\begin{itemize}
\item Use only one type of numbers: real numbers
\begin{itemize}
\item Chosen to be intuitive for studends at end of K12
\item Including the order relation
\end{itemize}
\item View other known types as subsets
\item Include stability laws in silent proof automation
\item Strong inspiration: the PVS approach
\begin{itemize}
\item However PVS is too aggressive on automation for education
\end{itemize}
\item Natural numbers, integers, etc, still silently present in the background
\end{itemize}
\end{frame}
\begin{frame}
\frametitle{Plan}
\begin{itemize}
\item Review of usages of natural numbers and integers
\item Defining subsets of \(\mathbb R\) for inductive types
\item From \(\mathbb Z\) and \(\mathbb N\) to \(\mathbb R\) and back
\item Ad hoc proofs of membership
\item Finite sets and big operations
\item Recursive definitions and iterated functions
\item Minimal set of tactics
\item Practical examples, around factorials and binomials
\end{itemize}
\end{frame}
\begin{frame}
\frametitle{Usages of natural numbers and integers}
\begin{itemize}
\item A basis for proofs by induction
\item iterating an operation a number of time (iterated derivatives)
\item The sequence \(0\ldots n\)
\item indices for finite collections, \(\sum_{i=m}^{n} f(i)\)
\item Recursive sequence definition
\item Specific to Coq+Lean+Agda: constructor normal forms as targets of
reduction
\item In Coq, numbers 0, 1, ..., 37, ... rely on integers for representation
\begin{itemize}
\item In Coq, you can define {\tt Zfact} as an efficient equivalent of factorial and compute 100!
\end{itemize}
\end{itemize}
\end{frame}
\begin{frame}
\frametitle{Defining subsets of \(\mathbb R\) for inductive types}
\begin{itemize}
\item Inductive predicate approach
\begin{itemize}
\item Inherit the induction principle
\item Prove the existence of a corresponding natural or integer
\end{itemize}
\item Existence approach
\begin{itemize}
\item Show the properties normally used as constructors
\item Transport the induction principle from the inductive type to the predicate
\end{itemize}
\end{itemize}
\end{frame}
\begin{frame}[fragile]
\frametitle{Inductive predicate in Coq}
\begin{alltt}
Require Import Reals.
Open Scope R_scope.

Inductive Rnat : R -> Prop :=
  Rnat0 : Rnat 0
| Rnat_succ : forall n, Rnat n -> Rnat (n + 1).
\end{alltt}
Generated induction principle:
\begin{alltt}
nat_ind
     : forall P : R -> Prop,
       P 0 ->
       (forall n : R, Rnat n -> P n -> P (n + 1)) ->
       forall r : R, Rnat r -> P r
\end{alltt}
\end{frame}
\begin{frame}[fragile]
\frametitle{Natural numbers as injections}
\begin{alltt}
Definition Rnat x : exists n, x = INR n.

Lemma  Rnat_add x y : Rnat x -> Rnat y -> Rnat (x + y).
Proof.
intros [n xn] [m ym]; exists (n + m)%nat.
now rewrite xn, ym, plus_INR.
Qed.
\end{alltt}
Key: use witnesses from definitions, then morphism laws.
\end{frame}
\begin{frame}
\frametitle{from \(\mathbb N\) and \(\mathbb Z\) to \(\mathbb R\) and back}
\begin{itemize}
\item Reminder: the types \(\mathbb N\) ({\tt nat}) and \(\mathbb Z\) ({\tt Z}),
should not be exposed
\item Injections {\tt INR} and {\tt IZR} already exist
\item New functions {\tt IRN} and {\tt IRZ}
\item definable using Hilbert's choice operator
\begin{itemize}
\item Requires {\tt ClassicalEpsilon}
\item use the inverse image for {\tt INR} and {\tt IZR} when {\tt Rnat}
or {\tt Rint} holds
\end{itemize}
\end{itemize}
\end{frame}
\begin{frame}[fragile]
\frametitle{Degraded typing}
\begin{itemize}
\item Stability laws provide automatable proofs of membership
\end{itemize}
\begin{alltt}
Lemma Rnat_add x y : Rnat x -> Rnat y -> Rnat (x + y).
Proof. ... Qed.

Lemma Rnat_mul x y : Rnat x -> Rnat y -> Rnat (x * y).
Proof. ... Qed.

Lemma Rnat_pos : Rnat (IZR (Z.pos _)).
Proof. ... Qed.

Hint Resolve Rnat_add Rnat_mul Rnat_pos : rnat.
\end{alltt}
\begin{itemize}
\item {\tt auto with rnat} will prove automatically
   {\tt Rnat x -> Rnat ((x + 2) * x)}.
\end{itemize}
\end{frame}
\begin{frame}
\frametitle{Ad hoc proofs of membership}
\begin{itemize}
\item When \(n, m \in {\mathbb N}, m < n\), \((n - m) \in {\mathbb N}\) can
also be proved
\item This requires an explicit proofs
\item Probably good in a training context for students
\item Similar for division
\end{itemize}
\end{frame}
\begin{frame}
\frametitle{Finite sets of indices}
\begin{itemize}
\item Usual mathematical idiom : \(1 \ldots n\), \(0 \ldots n\), \((v_i)_{i = 1 \ldots n}\)
\item Provide a {\tt Rseq : R -> R -> R}
\begin{itemize}
\item {\tt Rseq 0 3 = [0; 1; 2]}
\end{itemize}
\item Using the inductive type of lists here
\item This may require explaining structural recursive programming to students
\item At least {\tt map} and {\tt cat} (noted {\tt ++})
\end{itemize}
\end{frame}
\begin{frame}
\frametitle{Big sums and products}
\begin{itemize}
\item Taking inspiration from Mathematical components
\item {\tt \char'134{}big[op/idx]\_(a <= i < b) f(i)}
\begin{itemize}
\item Big sum when {\tt op = Rplus} and {\tt idx = 0}
\end{itemize}
\item Well-typed when {\tt a} and {\tt b} are real numbers\\
 (plus typing conditions on {\tt op}, {\tt idx}, and {\tt f})
\item Relevant when \({\tt a} < {\tt b}\)
\item This needs a hosts of theorems
\begin{itemize}
\item Chipping off terms at both ends
\item Cutting in the middle
\item Shuffling the indices
\end{itemize}
\item Mathematical Components {\tt bigop} library provides a guideline
\end{itemize}
\end{frame}
\begin{frame}[fragile]
\frametitle{Iterated functions}
\begin{itemize}
\item Mathematical idiom : \(f ^ n\), when \(f : A -> A\)
\item We provide {\tt Rnat\_iter} whose numeric argument is a real number
\item Only meaning full when the real number satisfies {\tt Rnat}
\item Useful to define many of the functions we are accustomed to see
\item Very few theorems are needed to explain its behavior
\begin{itemize}
\item \(f^{n+m}(a) = f^{n}(f^{m}(a))\quad f^1(a) = f(a)\quad f^0(a) = a\)
\end{itemize}
\end{itemize}
\end{frame}
\begin{frame}
\frametitle{Recursive functions}
\begin{itemize}
\item recursive sequences are also a typical introductory subject
\item As an illustration, let us consider the {\em Fibonacci} sequence\\
{\em The Fibonacci sequence is the function \(F\) such that \(F_0= 0\), \(F_1=1\), and \(F_{n+2} = F_{n} + F_{n + 1}\) for every natural number \(n\)}
\item Proof by induction and the defining equations are enough to
{\em study} a sequence
\item But {\em def{}ining} is still needed
\item Solution: define a recursive definition command using Elpi
\end{itemize}
\end{frame}
\begin{frame}[fragile]
\frametitle{Definition of recursive functions}
\begin{itemize}
\item We can use a {\em recursor}, mirror of the recursor on natural numbers
\item {\tt Rnat\_rec : ?A -> (R -> ?A -> ?A) -> R -> ?A}
\item Multi-step r�cursion can be implemented by using tuples of the
  right size
\end{itemize}
\begin{small}
\begin{alltt}
(* fib 0 = 0  fib 1 = 1              *)
(* fib n = fib (n - 1) + fib (n - 2) *)

Definition fibr := Rnat_rec [0; 1]
   (fun n l => [nth 1 l 0; nth 0 l 0 + nth 1 l 0]).
\end{alltt}
\end{small}
\end{frame}
\begin{frame}[fragile]
\frametitle{Meta-programming a recursive definition command}
\begin{itemize}
\item The definition in the previous slide can be generated
\item Taking as input the equations (in comments)
\item The results of the definition are in two parts
\begin{itemize}
\item The function of type {\tt R -> R}
\item The proof the logical statement for that function
\end{itemize}
\end{itemize}
\begin{small}
\begin{alltt}
Recursive (def fib such that
             fib 0 = 0 /\
             fib 1 = 1 /\
             forall n : R, Rnat (n - 2) ->
                fib n = fib (n - 2) + fib (n - 1)).
\end{alltt}
\end{small}
\end{frame}
\begin{frame}
\frametitle{Minimal set of tactics}
\begin{itemize}
\item {\tt replace}
\begin{itemize}
\item {\tt ring} and {\tt field} for justifications
\item No need to massage formulas step by step through rewriting
\end{itemize}
\item {\tt intros}, {\tt exists}, {\tt split}, {\tt destruct} to handle
  logical connectives (as usual)
\item {\tt rewrite} to handle the behavior of all defined functions
 (and recursors)
\item {\tt unfold} for functions defined by students
\item {\tt apply} and {\tt lra} to handle all side conditions related to bounds
\item {\tt typeclasses eauto} to prove membership in {\tt Rnat}
\begin{itemize}
\item Explicit handling for subtraction and division
\end{itemize}
\item Possibility to add ad-hoc computing facilities for user-defined
\begin{itemize}
\item Relying on mirror functions computing on inductive {\tt nat} or {\tt Z}
\end{itemize}
\end{itemize}
\end{frame}
\begin{frame}
\frametitle{Demonstration time}
\begin{itemize}
\item A study of factorials and binomial numbers
\begin{itemize}
\item Efficient computation of factorial numbers
\item Proofs relating the two points of view on binomial numbers,
   ratios or recursive definition
\item A proof of the expansion of \((x + y) ^ n\)
\end{itemize}
\item A study the fibonacci sequence
\begin{itemize}
\item \({\cal F}(i) = \frac{\phi ^ i - \psi ^ i}{\phi - \psi}\) (\(\phi\) golden ratio)
\end{itemize}
\end{itemize}
\end{frame}
\begin{frame}
\frametitle{The ``17'' exercise}
\begin{itemize}
\item Prove that there exists an \(n\) larger than 4 such that\\
\[\left(\begin{array}{c}n\\5\end{array}\right) =
  17 \left(\begin{array}{c}n\\4\end{array}\right)\]\\
(suggested by S. Boldo, F. Cl�ment, D. Hamelin, M. Mayero, P. Rousselin)
\item Easy when using the ratio of factorials and eliminating common
sub-expressions on both side of the equality\\
\[\frac{\cancel{n!}}{\cancel{(n - 5)!}\cancel{5!}_5} = 17 \frac{\cancel{n!}}{\cancel{(n-4)!}_{(n-4)}\cancel{4!}}\]
\item They use the type of natural numbers and equation\\
\[\left(\begin{array}{c}n\\p + 1\end{array}\right) \times (p + 1) =
  \left(\begin{array}{c}n\\p\end{array}\right) \times (n - p)\]\\
\end{itemize}
\end{frame}
\end{document}


%%% Local Variables: 
%%% mode: latex
%%% TeX-master: t
%%% End: 
